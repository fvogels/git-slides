% IGNORE

\section{Version Control}

\begin{frame}
  \tableofcontents[currentsection]
\end{frame}

\begin{frame}
  \frametitle{Problem Statement}
  \begin{itemize}
    \item Modifying code is risky
    \item We might break something
    \item Keep a fully functional version somewhere safe (= a backup)
    \item In case of trouble, we can simply restore this backup
  \end{itemize}
\end{frame}

\begin{frame}
  \frametitle{Manual Solution}
  \begin{center}
    \begin{tikzpicture}
      \path[use as bounding box] (0,0) rectangle (3,-3);
      \directory[id=root,position={(0,0)}]{project}
      \directory[id=working-area,subdirectory of=root distance 1]{working-area}
      \directory[id=wa-src,subdirectory of=working-area distance 1]{src}
      \directory[id=wa-app,subdirectory of=wa-src distance 1]{app.py}
      \directory[id=wa-lib,subdirectory of=wa-src distance 2]{lib.py}
      \directory[id=backup,subdirectory of=root distance 5]{backup}

      \only<3->{
        \directory[id=bu-src,subdirectory of=backup distance 1]{src}
        \directory[id=bu-app,subdirectory of=bu-src distance 1]{app.py}
        \directory[id=bu-lib,subdirectory of=bu-src distance 2]{lib.py}
      }

      \draw[fs/dirlink] (root) |- (backup);
      \draw[fs/dirlink] (root) |- (working-area);
      \draw[fs/dirlink] (working-area) |- (wa-src);
      \draw[fs/dirlink] (wa-src) |- (wa-app);
      \draw[fs/dirlink] (wa-src) |- (wa-lib);

      \only<3->{
        \draw[fs/dirlink] (backup) |- (bu-src);
        \draw[fs/dirlink] (bu-src) |- (bu-app);
        \draw[fs/dirlink] (bu-src) |- (bu-lib);
      }

      \only<1>{
        \draw[comment line,latex-] (working-area-node.east) -| ++(1,0.75) node[comment box,anchor=south] (comment) {Develop code in this folder};
      }
      \only<2>{
        \draw[comment line,latex-] (backup-node.east) -| ++(2,-0.5) node[comment box,anchor=north] {Regularly copy code into here};
      }
      \only<3>{
        \draw[copy arrow] (working-area-node.east) -- ++(1,0) |- (backup-node.east) node[midway,below] {copy};
      }
    \end{tikzpicture}
  \end{center}
  \vskip1cm
  \begin{overprint}
    \onslide<3>
    \begin{center}
      \begin{tikzpicture}
        \node[console] {
          \parbox{8cm}{
            \# Inside working-area \\
            \$ rm -r ../backup \\
            \$ cp -r * ../backup
          }
        };
      \end{tikzpicture}
    \end{center}
  \end{overprint}
\end{frame}

\begin{frame}
  \frametitle{Problem Statement}
  \begin{itemize}
    \item We don't just want backup of last working version
    \item We'd like to keep track of \emph{all} previous versions
    \item We need a separate directory per backup
  \end{itemize}
\end{frame}

\begin{frame}
  \frametitle{Manual Solution}
  \begin{center}
    \begin{tikzpicture}
      \path[use as bounding box] (0,0) rectangle (3,-2);
      \directory[id=root,position={(0,0)}]{project}
      \directory[id=working-area,subdirectory of=root distance 1]{working-area}
      \directory[id=backups,subdirectory of=root distance 2]{backups}

      \draw[fs/dirlink] (root) |- (backups);
      \draw[fs/dirlink] (root) |- (working-area);

      \only<2->{
        \directory[id=backup1,subdirectory of=backups distance 1]{20220329-1531}
        \draw[fs/dirlink] (backups) |- (backup1);
      }

      \only<2>{
        \draw[copy arrow] (working-area-node.east) -- ++(1,0) |- (backup1-node.east);
      }

      \only<3->{
        \directory[id=backup2,subdirectory of=backups distance 2]{20220330-1022}
        \draw[fs/dirlink] (backups) |- (backup2);
      }

      \only<3>{
        \draw[copy arrow] (working-area-node.east) -- ++(1,0) |- (backup2-node.east);
      }

      \only<4->{
        \directory[id=backup3,subdirectory of=backups distance 3]{20220331-1503}
        \draw[fs/dirlink] (backups) |- (backup3);
      }

      \only<4>{
        \draw[copy arrow] (working-area-node.east) -- ++(1,0) |- (backup3-node.east);
      }

    \end{tikzpicture}
  \end{center}
  \vskip1cm
  \begin{overprint}
    \onslide<2-4>
    \begin{center}
      \begin{tikzpicture}
        \node[console] {
          \parbox{8cm}{
            \# Inside working-area \\
            \$ DIR=../backups/\`{}date +\%Y\%m\%d-\%H\%m\`{} \\
            \$ mkdir -p \$DIR \\
            \$ cp -r * \$DIR
          }
        };
      \end{tikzpicture}
    \end{center}
  \end{overprint}
\end{frame}

\begin{frame}
  \frametitle{Git To The Rescue}
  \begin{itemize}
    \item Git automates this process
    \item Git is smart about storage
      \begin{itemize}
        \item Will not duplicate files needlessly
        \item Uses delta compression
      \end{itemize}
  \end{itemize}
\end{frame}

\begin{frame}
  \frametitle{Git Modus Operandi}
  \begin{center}
    \begin{tikzpicture}
      \directory[id=root,position={(0,0)}]{project}
      \directory[id=git,subdirectory of=root distance 1]{.git}
      \directory[id=src,subdirectory of=root distance 2]{src}
      \directory[id=app,subdirectory of=src distance 1]{app.py}
      \directory[id=lib,subdirectory of=src distance 2]{lib.py}

      \draw[fs/dirlink] (root) |- (git);
      \draw[fs/dirlink] (root) |- (src);
      \draw[fs/dirlink] (src) |- (app);
      \draw[fs/dirlink] (src) |- (lib);

      \visible<4->{
        \node[git/commit] (commit v1) at (0,-3) {v1};
      }
      \visible<7->{
        \node[git/commit] (commit v2) at ($ (commit v1) + (2,0) $) {v2};
        \draw[git/commit arrow] (commit v1) -- (commit v2);
      }

      \begin{scope}[overlay]
        \only<1>{
          \draw[comment line,latex-] (root-node.east) -| ++(1,0.5) node[comment box,anchor=south] {You work directly in project's root directory};
        }
        \only<2>{
          \draw[comment line,latex-] (git-node.east) -| ++(1,1) node[comment box,anchor=south] {Git stores data in hidden directory};
        }
        \only<3>{
          \draw[comment line,latex-] (app-node.east) -| ++(1,-1) node[comment box,anchor=north] (explanation) {\parbox{5cm}{First, you must let Git know which files should be backed up}};
          \draw[comment line,latex-] (lib-node.east) -| (explanation.north);
        }
        \only<4>{
          \draw[comment line,latex-] (git-node.east) -| ++(1.5,0.75) node[comment box,anchor=south] {\parbox{5cm}{\emph{Committing} tells Git to take a snapshot and store it internally}};
          \draw[comment line,latex-] (commit v1.west) -| ++(-1.25,0.75) node[comment box,anchor=south] {\parbox{3cm}{We represent a commit by a circle}};
        }
        \only<5>{
          \draw[comment line,latex-] (app-node.east) -| ++(1,-1) node[comment box,anchor=north] (explanation) {Say you update this file};
        }
        \only<6>{
          \draw[comment line,latex-] (app-node.east) -| ++(1,-1) node[comment box,anchor=north] (explanation) {\parbox{5cm}{We need to tell Git explicitly which file changes to commit}};
        }
        \only<7>{
          \draw[comment line,latex-] (commit v2.east) -| ++(1,-0.5) node[comment box,anchor=north] {A new commit is added};
        }
      \end{scope}
    \end{tikzpicture}
  \end{center}
  \vskip1cm
  \begin{overprint}
    \onslide<3>
    \begin{center}
      \begin{tikzpicture}
        \node[console] {
          \parbox{8cm}{
            \$ git add src/*.py
          }
        };
      \end{tikzpicture}
    \end{center}

    \onslide<4>
    \begin{center}
      \begin{tikzpicture}
        \node[console] {
          \parbox{8cm}{
            \$ git commit -m 'v1'
          }
        };
      \end{tikzpicture}
    \end{center}

    \onslide<6>
    \begin{center}
      \begin{tikzpicture}
        \node[console] {
          \parbox{8cm}{
            \$ git add src/app.py
          }
        };
      \end{tikzpicture}
    \end{center}

    \onslide<7>
    \begin{center}
      \begin{tikzpicture}
        \node[console] {
          \parbox{8cm}{
            \$ git commit -m 'v2'
          }
        };
      \end{tikzpicture}
    \end{center}
  \end{overprint}
\end{frame}

\begin{frame}
  \frametitle{Commit Chain}
  \begin{center}
    \begin{tikzpicture}[commit data/.style={draw}]
      \foreach[evaluate={-1.5*(\i-1)} as \y] \i in {1,...,4} {
        \coordinate (commit position \i) at (0,\y);
      }

      \node[git/commit] (commit 1) at (commit position 1) {};

      \visible<1-4>{
        \node[comment box,overlay] (comment) at ($ (commit position 1) ! 0.5 ! (commit position 4) + (4,0) $) {\parbox{4cm}{New commits are made as project progresses}};
      }
      \visible<1>{
        \draw[comment line,latex-,overlay] (commit 1.east) -| (comment);
      }

      \foreach[evaluate={int(\i-1)} as \j,evaluate={\i} as \slideindex] \i in {2,...,4} {
        \visible<\slideindex->{
          \node[git/commit] (commit \i) at (commit position \i) {};
          \draw[git/commit arrow] (commit \j) -- (commit \i);
        }
        \visible<\slideindex>{
          \draw[comment line,latex-] (commit \i.east) -| (comment);
        }
      }

      \visible<5->{
        \node[anchor=west,font=\tiny,commit data] (commit data 1) at ($ (commit 1) + (0.25,0) $) {
          \parbox{6cm}{
            \begin{tabular}{ll}
              \textbf{Message} & Added sound effects \\
              \textbf{Author} & John \\
              \textbf{Date} & 24 March 2022 \\
              \textbf{Time} & 15:25 \\
            \end{tabular}
          }
        };

        \node[anchor=west,font=\tiny,commit data] (commit data 2) at ($ (commit 2) + (0.25,0) $) {
          \parbox{6cm}{
            \begin{tabular}{ll}
              \textbf{Message} & Increased spaceship speed \\
              \textbf{Author} & Peter \\
              \textbf{Date} & 24 March 2022 \\
              \textbf{Time} & 15:31 \\
            \end{tabular}
          }
        };

        \node[anchor=west,font=\tiny,commit data] (commit data 3) at ($ (commit 3) + (0.25,0) $) {
          \parbox{6cm}{
            \begin{tabular}{ll}
              \textbf{Message} & Optimized collision detection algorithm \\
              \textbf{Author} & Anna \\
              \textbf{Date} & 24 March 2022 \\
              \textbf{Time} & 16:11 \\
            \end{tabular}
          }
        };

        \node[anchor=west,font=\tiny,commit data] (commit data 4) at ($ (commit 4) + (0.25,0) $) {
          \parbox{6cm}{
            \begin{tabular}{ll}
              \textbf{Message} & Changed text color in introduction \\
              \textbf{Author} & Peter \\
              \textbf{Date} & 24 March 2022 \\
              \textbf{Time} & 17:44 \\
            \end{tabular}
          }
        };
      }

      \visible<5>{
        \begin{scope}[overlay]
          \node[comment box] (comment) at ($ (commit data 1.north east) + (0.25,0.25) $) {\parbox{3.5cm}{Git stores metadata about each commit}};
          \draw[comment line,-latex] (comment.south) |- (commit data 1.east);
          \draw[comment line,-latex] (comment.south) |- (commit data 2.east);
          \draw[comment line,-latex] (comment.south) |- (commit data 3.east);
          \draw[comment line,-latex] (comment.south) |- (commit data 4.east);
        \end{scope}
      }

      \visible<6>{
        \node[comment box] at ($ (commit data 4.south east) $) {\parbox{3.5cm}{Each commit represents a small update}};
      }
    \end{tikzpicture}
  \end{center}
\end{frame}
